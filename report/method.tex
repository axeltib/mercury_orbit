\section{Method}\label{sec:method}

\subsection{Integrators}

Before describing the integrators, we introduce the following notation for the state vector

\begin{equation}
	(x_1, x_2, x_3, x_4) := (r, \dot{r}, \phi, \dot{\phi})
\end{equation}

and is denoted by $\textbf{x}^n$, where $i$ denotes the current step. Furthermore, the function $\textbf{f}(\textbf{x})$ is then defined as the EOMs \eqref{eq:eoms}. Finally, the timestep is defined as $h$.  

\subsubsection{Euler method}
Being one of the simplest numerical integrators, it serves as a good foundation to compare the other methods to. As we will se does it behave better in some regards, but is generally found to be the lower limit of performance between the integrators. The equation defining this method is

\begin{equation}\label{eq:euler-method-equation}
	\textbf{x}^{n+1} = \textbf{x}^n + h \textbf{f}(\textbf{x}^n) 
\end{equation}


\subsubsection{Backward Euler method}

Because this method depends on the equations of motion and the derivative (Jacobian) of the as well, it was much more difficult to implement, and in the case that the EOM are slightly changed, one must recalculate the derivative, and hence reformulate the algorithm for Euler backward as well. This makes this method slightly inconvinient to use, but in the case, such as here, where only one system of equations are investigated, it is adequate practically. 

Just as in 1 dimesnion, Backward Euler here depends on Newton's method to solve for the next step. Here we have 4 dimensions, and the equation to be solved is 

\begin{equation}
	\mathbf{F}(\mathbf{x}^{n+1}) = \mathbf{x}^{{n+1}} - h \mathbf{f}(\mathbf{x}^{{n+1}}) - \mathbf{x}^n = 0
\end{equation}

To solve this equation, Newton's method must be employed. In order to use this method, the vector correspondant to the derivative must be used: the Jacobian. The algorithm is as follows: 

\begin{enumerate}
	\item Initialize $\textbf{x}^{n+1}_0$ with a good guess: pick $\textbf{x}^n$
	\item Calculate $\textbf{F} = \textbf{x}^{n+1}_{j} - \textbf{x}^{n} - h \textbf{f}(\textbf{x}^{n+1}_{j})$
	\item Calcuate the Jacobian $\mathcal{J}$ at the point $\textbf{x}^{n+1}_{j}$
	\item $\textbf{x}^{n+1}_{j+1} = \textbf{x}^{n+1}_{j} - \mathcal{J}^{-1} \textbf{F}$
	\item If $|\textbf{x}^{n+1}_{j+1} - \textbf{x}^{n+1}_{j}| \le $ threshold: next step is found.
	\item Else: return to step 2.
\end{enumerate}

where $j$ here is the step of the Newton's method. Note here that $t$ only depends on itself, and will thus not be handled in the numerical methods, as it is trivial.  

\subsubsection{Runge-Kutte 4 (RK4)}

Runge Kutta 4 is sort of an extension of the Euler Method, rather than calculating the timestep in one interval, RK4 does it in 4 steps. The method has the following equation:


\begin{align}
	\textbf{x}^{n+1} &= \frac{1}{6}h(k_1 + 2k_2 + 2k_3 + k_4) \textbf{x}^n \\
	k_1 &= \textbf{f}(\textbf{x}^n) \\
	k_2 &= \textbf{f}(\textbf{x}^n + \frac{1}{2}h k_1) \\
	k_3 &= \textbf{f}(\textbf{x}^n + \frac{1}{2}h k_2) \\
	k_4 &= \textbf{f}(\textbf{x}^n + h k_3)
\end{align}

The structure of the Euler method can be seen, but divided up into 4 smaller steps and the result is a weighted average over them. 

\subsection{Analysis}

\subsubsection{Energy conservation}
To study how the integrators compare to each other we will investigate three properties. In orbital mechanics, and all dynamic systems without friction, it is important to conserve the energy. A system that looses energy will be less valid compared to one that does not when solving systems numerically. Energy conservation is the first thing that will be investigated. Given in equation (18.34) \cite{gould_2007}, energy per unit mass is given by

\begin{equation}\label{eq:system-energy}
	e = \left(1-\frac{2M}{r}\right)\frac{d\tau}{dt}
\end{equation}

where

\begin{equation}
	\frac{d\tau}{dt} = \left[\left(1-\frac{2M}{r}\right) - \left(1-\frac{2M}{r}\right)^{-1} \dot{r}^2 - r^2 \dot{\phi}^2\right]^{-1/2}	
\end{equation}

With this can the energy conservation be studied. In order to do that must initial values be set up. To not make matters too complicated will we study a system which is a small perturbance from the circular orbit solution, which is given by 

\begin{equation}
	v = \sqrt{M/r}, \; r \geq 6M
\end{equation}

To it we add the perturbation (INSERT PERTURBATION HERE). 

\subsubsection{Error compared to analytical solution - RMSE}

\subsubsection{System dependence}
Might be stupid tos study hahah





